\documentclass[a4paper,10pt]{article}
%\usepackage[latin1]{inputenc} % Paquetes de idioma (otro encoding)
\usepackage[utf8]{inputenc} % Paquetes de idioma
\usepackage[spanish]{babel} % Paquetes de idioma
\usepackage{graphicx} % Paquete para ingresar gráficos
\usepackage{grffile}
\usepackage{hyperref}
\usepackage{fancybox}
\usepackage{amsmath}
\usepackage{amsfonts}
\usepackage{listings}
\usepackage{pdfpages}
% Paquetes de macros de Circuitos
%\usepackage{pstricks}
\usepackage{tikz}

% Encabezado y Pié de página
% Paquete para encabezados y pie de página
\usepackage{fancyhdr} 
% Sin esta línea no se imprimiría el encabezado en todas las páginas
\pagestyle{fancy} 

% Borra el encabezado anterior (Por defecto escribe el título de la 
%  sección en la que se encuentra la hoja
\fancyhf{} 
\setlength{\headheight}{22.55pt}
\fancyhead[L]{
	{\textsf{Facultad de Ingeniería $-$ Universidad de Buenos Aires 
    \\ 75.61 Taller de Programación III}}
}
%\addtocounter{page}{5}
\fancyhead[R]{\thepage}

% Ajusta el tamaño de las líneas separadoras en el pié de página
\renewcommand{\footrulewidth}{0.4pt}
 % Ajusta el tamaño de las líneas separadoras en el encabezado
\renewcommand{\headrulewidth}{0.4pt} 

\fancyfoot[L]{
	{TP N$^{\circ}$3 - Event Managment (Google App Engine)}\\
	{\textsf{Integrantes: Torres Feyuk}}
}
		

% Carátula del Trabajo
\title{ \input{Portada.tex} }

\begin{document}
	\maketitle % Hace que el título anterior sea el principal del documento
	\newpage

    % Esta línea genera un indice a partir de las secciones y 
    % subsecciones creadas en el documento
	\tableofcontents 
	\newpage

	\section{Introducción}
		El presente trabajo práctico consiste en diseñar e implementar un sistema de recepción de eventos a través
        de la plataforma de cloud computing de google \href{http://cloud.google.com/appengine}{App Engine}. El
        presente sistema debe permitir el ingreso de personas invitadas en diferentes eventos, como asi también
        tener la posibilidad de poder consultar quienes son los invitados anotados en cada uno de ellos.
          
         
    \section{Desarrollo}
        Google App Engine permite realizar el desarrollo de las aplicaciones web a través de los siguientes 
        lenguajes de programación:

        \begin{itemize}
            \item Java
            \item Python
            \item Go
            \item PHP
        \end{itemize}

        Debido a la simplicidad del lenguaje y la falta de conocimientos en las otras opciones, se decidió implementar
        el sistema en Python. App Engine permite realizar el desarrollo de la parte Servidor a través de cualquier
        web framework de Python que implemente la interfaz WSGI(Web Service Gateway Interface). Se lista a continuación
        algunos de los frameworks que cumplen con esta especificación:
        
        \begin{itemize}
            \item Django
            \item CherryPy
            \item Pylon 
            \item web.py
            \item web2pyi
            \item webapp2
        \end{itemize}

        Para realizar la presente aplicación se decidió utilizar \text{webapp2}. El motivo de esta elección es que
        la complejidad del presente proyecto era baja y los tutoriales de google respecto a las aplicaciones de 
        Python para App Engine están desarrolladas con este framework. \\
        \indent Para el desarrollo y diseño de la página web se utilizó HTML, scripts en Javascript y JQuery 
        y el template language Jinja2. Este último fue utilizado la generación estática del página principal del 
        proyecto, la cual nuevamente se realizó en función del 
        \href{https://cloud.google.com/appengine/docs/python/gettingstartedpython27/introduction}{tutorial de App Engine}.
        

    \newpage
    \section{Arquitectura 4 + 1}

    \newpage
    \subsection{Casos de Uso}

    \newpage
    \subsection{Vista Lógica}

    \newpage
    \subsection{Vista de Actividad}

    \newpage
    \subsection{Vista de Despliegue}

    \newpage
    \section{Resultados}
        TODO:

    \newpage
    \subsection{Pruebas de Carga}

    \newpage
    \section{Conclusión} 
        TODO:

\end{document}
        
      
